\documentclass{beamer}

\usepackage{graphicx}
\usepackage{animate}
\usepackage{bm,amsmath,amssymb,amsfonts}
\usepackage{multirow, booktabs}


\usepackage{lmodern} 
\usefonttheme{professionalfonts}



\renewcommand{\vec}[1]{\bm{#1}}


\usepackage[utf8]{inputenc}



\usetheme{Madrid}
\usecolortheme{dolphin}





\title{Robust ORCA: Ellipsoidal Uncertainty}
\author{Zhaoli Wang \and Guxin Du \and Yiran Yang}
\date{\today}

\begin{document}

% ------------------------------------------------------
\frame{\titlepage}
% ------------------------------------------------------

\section{MAPP}

\begin{frame}{Multi-Agent Path Planning (MAPP)}
\textbf{Problem setting:}
\begin{itemize}
    \item We consider $N$ agents moving in a shared 2D workspace.
    \item Each agent chooses a velocity and then moves in a 
          \textbf{straight line} at constant speed.
    \item After a fixed time step, each agent updates its velocity again.
    \item The key requirement: \textbf{agents must avoid collisions} 
          while moving toward their goals.
\end{itemize}

\vspace{2mm}
\textbf{Goal:}
\begin{itemize}
    \item Enable all agents to reach their destinations efficiently
          \textbf{without colliding}.
\end{itemize}
\end{frame}




% ------------------------------------------------------
\begin{frame}{Key Challenges in MAPP}
\small
\textbf{1. Multi-stage decision making}
\begin{itemize}
    \item Agents repeatedly select velocities and update their positions.
    \item Decisions across time steps are tightly coupled.
\end{itemize}

\vspace{2mm}
\textbf{2. Centralized coupling}
\begin{itemize}
    \item A centralized solver must choose velocities for all agents jointly.
    \item The number of pairwise collision constraints scales as $O(N^2)$.
\end{itemize}

\vspace{2mm}
\textbf{3. Non-convex feasible region}
\begin{itemize}
    \item Collision avoidance requires
    \[
        \|\vec p_i + t\,\vec v_i 
          - (\vec p_j + t\,\vec v_j)\|_2 
        \;\ge\; r_i + r_j,
        \quad \forall\,  t \in [t_{cur}, t_{cur} + \tau].
    \]
    \item The feasible set of $\vec v_i$ is the complement of a disk
          in velocity space $\Rightarrow$ non-convex.
\end{itemize}

\vspace{2mm}
\textbf{4. Uncertainty in real systems}
\begin{itemize}
    \item Sensing and actuation noise make future positions uncertain.
\end{itemize}

\end{frame}
% ------------------------------------------------------






\section{ORCA}


% ------------------------------------------------------

% ------------------------------------------------------
\begin{frame}{ORCA: Advantages and Applications}
\textbf{Key advantages}
\begin{itemize}
    \item \textbf{Fully decentralized}: each agent selects its own velocity using only local neighbor information.
    \item \textbf{One-step decision making}: at every time step ORCA chooses a velocity for the next horizon, without planning a full trajectory.
    \item \textbf{Real-time performance}: scales to thousands of agents at interactive frame rates.
\end{itemize}

\vspace{0.4cm}
\textbf{Application domains}
\begin{itemize}
    \item Video games and animation (e.g., crowd scenes in \emph{Warhammer 40,000: Space Marine} and other AAA titles).
    \item Pre-training / guiding RL policies for crowd and robot navigation.
    \item Large-scale, high-density scenarios
\end{itemize}
\end{frame}

% ------------------------------------------------------
\begin{frame}{ORCA: Basic Idea}
\begin{itemize}
    \item At each decision step and for each neighbor $j$, define a velocity obstacle
    \[
        \mathrm{VO}_{i|j}^\tau
        :=
        \bigl\{\, v_i-v_j \in \mathbb{R}^2
        \;\big|\;\text{ leads to collision within horizon }\tau
        \bigr\}.
    \]
    \item Relax $\mathrm{VO}_{i|j}^\tau$ into a separating half-space, and then,
      by the reciprocal protocol, assign half of this avoidance responsibility
      to agent $i$ as $\mathrm{ORCA}_{i|j}^\tau$.
    \item The feasible velocity region for agent $i$ is the intersection
    \[
        \mathcal{V}_i^\tau
        =
        \bigcap_{j \in \mathcal{N}_i} \mathrm{ORCA}_{i|j}^\tau.
    \]
    \item Update velocity by projecting the preferred velocity onto this feasible set:
    \[
        v_i^{\text{new}}
        =
        \arg\min_{v \in \mathcal{V}_i^\tau}
        \| v - v_i^{\text{pref}} \|_2^2.
    \]
\end{itemize}
\end{frame}
% ------------------------------------------------------

\begin{frame}{ORCA: Feasible Region Relaxation}
  \centering
  \includegraphics[width=0.8\textwidth]{1.jpg}
\end{frame}


% ------------------------------------------------------
\begin{frame}{ORCA: Optimization Model}

\textbf{Optimization formulation for agent $i$:}
\[
\begin{aligned}
    \min_{\vec v \in \mathbb{R}^2}\quad
        & \|\vec v - \vec v_i^{\mathrm{pref}}\|_2^2 \\
    \text{s.t.}\quad
        & (\vec v - (\vec v_i + \tfrac12\, \vec u_{i|j}))^\top 
          \vec n_{i|j} \;\ge\; 0,
          \qquad \forall j \in \mathcal N_i, \\
        & \|\vec v\|_2 \le v_i^{\max}.
\end{aligned}
\]

\vspace{4mm}
\textbf{Uncertainty model:}
\begin{itemize}
    \item The observed positions and velocities depend on a 
          \textbf{joint noise vector}
    \[
        \boldsymbol{\xi} = (\xi_1, \xi_2, \dots, \xi_{|\mathcal{N}_i|}).
    \]

    \item The ORCA displacement and normal depend on this joint noise:
    \[
        \vec u_{i|j} = \vec u_{i|j}(\boldsymbol{\xi}),
        \qquad
        \vec n_{i|j} = \vec n_{i|j}(\boldsymbol{\xi}).
    \]
\end{itemize}

\end{frame}
% ------------------------------------------------------



% % ------------------------------------------------------
\section{Stochastic Optimization Models}

\begin{frame}{Stochastic Model 1: Minimizing the Expected Loss}

We model the ORCA update as a stochastic program:
\[
    z^\ast
    =
    \min_{\vec v \in X}
    \; \mathbb{E}_{\boldsymbol{\xi}}
    \bigl[\, f(\vec v, \boldsymbol{\xi}) \,\bigr].
\]

\vspace{3mm}
\textbf{Random loss function:}
\[
f(\vec v, \boldsymbol{\xi})
=
\begin{cases}
    \|\vec v - \vec v_i^{\mathrm{pref}}\|_2^2,
    & \text{if ORCA constraints hold for all neighbors } j, \\[1mm]
    +\infty,
    & \text{otherwise.}
\end{cases}
\]

\vspace{3mm}
\textbf{Feasible set:}
\[
X = \{\, \vec v \in \mathbb{R}^2 : \|\vec v\|_2 \le v_i^{\max} \,\}.
\]

\end{frame}
% % ------------------------------------------------------

% ------------------------------------------------------
\begin{frame}{Stochastic Model 2: Robust Optimization}

\textbf{Uncertainty set:}
\[
    \boldsymbol{\xi} \in \Xi.
\]

\vspace{3mm}
\textbf{Robust ORCA formulation for agent $i$:}
\[
\begin{aligned}
    \min_{\vec v \in \mathbb{R}^2} \quad 
        & \|\vec v - \vec v_i^{\mathrm{pref}}\|_2^2 \\
    \text{s.t.} \quad
        & (\vec v - (\vec v_i + \tfrac12\, \vec u_{i|j}(\boldsymbol{\xi})))^\top 
          \vec n_{i|j}(\boldsymbol{\xi})
          \;\ge\; 0, \\
        & \qquad\qquad\qquad
          \forall\, j \in \mathcal{N}_i,\;\forall\, \boldsymbol{\xi} \in \Xi, \\
        & \|\vec v\|_2 \le v_i^{\max}.
\end{aligned}
\]

\vspace{4mm}
\textbf{Interpretation:}
\begin{itemize}
    \item The chosen velocity must remain collision-free
          \textbf{for all} noise realizations in the uncertainty set $\Xi$.
    \item Guarantees worst-case safety.
\end{itemize}

\end{frame}
% ------------------------------------------------------

% ------------------------------------------------------
% \begin{frame}{Stochastic Model 3: Chance-Constrained Optimization}

% \[
% \begin{aligned}
%     \min_{\vec v \in \mathbb{R}^2}\quad
%         & \|\vec v - \vec v_i^{\mathrm{pref}}\|_2^2 \\
%     \text{s.t.}\quad
%         & \mathbb{P}\Big(
%             (\vec v - (\vec v_i + \tfrac12\, \vec u_{i|j}(\boldsymbol{\xi})))^\top
%             \vec n_{i|j}(\boldsymbol{\xi})
%             \ge 0
%           \Big)
%           \;\ge\; 1 - \alpha_j,
%           \quad \forall j \in \mathcal N_i,
%           \\[2mm]
%         & \textbf{or} \\[1mm]
%         & \mathbb{P}\Big(
%             (\vec v - (\vec v_i + \tfrac12\, \vec u_{i|j}(\boldsymbol{\xi})))^\top
%             \vec n_{i|j}(\boldsymbol{\xi})
%             \ge 0,\;\forall j \in \mathcal N_i
%           \Big)
%           \;\ge\; 1 - \alpha,
%           \\[2mm]
%         & \|\vec v\|_2 \le v_i^{\max}.
% \end{aligned}
% \]

% \end{frame}
% ------------------------------------------------------

% ------------------------------------------------------
\begin{frame}{Approximation Method: Sample Average Approximation}

Given $N$ sampled noise realizations 
$\{\boldsymbol{\xi}^{(1)}, \dots, \boldsymbol{\xi}^{(N)}\}$,

\vspace{2mm}
\textbf{SAA formulation:}
\[
\min_{\vec v \in \mathbb{R}^2}
    \frac{1}{N}
    \sum_{k=1}^{N}
    f\bigl(\vec v, \boldsymbol{\xi}^{(k)}\bigr)
\]

\vspace{3mm}
\textbf{Explicit SAA for ORCA:}
\[
\begin{aligned}
    \min_{\vec v \in \mathbb{R}^2}\quad
        & \frac{1}{N}
          \sum_{k=1}^N 
          \|\vec v - \vec v_i^{\mathrm{pref}}\|_2^2 \\
    \text{s.t.}\quad
        & (\vec v - (\vec v_i + \tfrac12\, 
            \vec u_{i|j}(\boldsymbol{\xi}^{(k)})))^\top 
          \vec n_{i|j}(\boldsymbol{\xi}^{(k)})
          \;\ge\; 0, \\
        & \qquad\qquad\forall j \in \mathcal N_i,\;\forall k = 1,\dots,N, \\
        & \|\vec v\|_2 \le v_i^{\max}.
\end{aligned}
\]

\vspace{3mm}
\textbf{Notes:}
\begin{itemize}
    \item Converts the stochastic problem into a deterministic but larger QP.
    \item Constraints scale linearly with sample size $N$.
\end{itemize}

\end{frame}
% ------------------------------------------------------


\section{Robust ORCA}


\begin{frame}{A Closer Look at the VO}
  \centering
  \includegraphics[width=0.8\textwidth]{3.jpg}
\end{frame}

% ------------------------------------------------------

\begin{frame}{A Closer Look at the VO}
  \centering
  \includegraphics[width=0.5\textwidth]{2.jpg}
\end{frame}


% ------------------------------------------------------
\begin{frame}{Robust Optimization under Ellipsoidal Velocity Noise}

\textbf{Uncertainty model:}
\[
    \tilde{\bm v}_j = \bar{\bm v}_j + \bm Q_j \xi_j,
    \qquad \|\xi_j\|_2 \le \Gamma_j .
\]

\medskip

When the closest point on the VO boundary lies on the extreme ray $\bm r_1$
(rather than on the lower-right circular cap), the robust ORCA constraint for 
agent $i$ w.r.t.\ neighbor $j$ is:

\[
    \bigl(
        \bm v - (\bm v_i + \tfrac12\, \bm u_{i|j}(\xi_j))
    \bigr)^\top
    \bm n_{i|j}
    \;\ge\; 0,
    \qquad
    \forall\, \|\xi_j\|_2 \le \Gamma_j .
\]

\medskip
\textbf{This can be written in deterministic closed form as:}
\[
    \bm n_{i|j}^\top \bm v
    \;\ge\;
    \bm n_{i|j}^\top \bm v_i
    + \tfrac12 \bm n_{i|j}^\top (\bm r_1 \bm r_1^\top - I)(\bm v_i - \bar{\bm v}_j)
    + \tfrac12\,\Gamma_j
      \bigl\| \bm Q_j^\top (I - \bm r_1 \bm r_1^\top)\, \bm n_{i|j} \bigr\|_2.
\]

\end{frame}
% ------------------------------------------------------

% ------------------------------------------------------
\begin{frame}{Preliminaries: Affine Form of $\bm u_{i|j}(\xi_j)$}
\small
\textbf{Velocity noise model.}
\[
    \tilde{\bm v}_j 
        = \bar{\bm v}_j + \bm Q_j \xi_j,
    \qquad 
    \|\xi_j\|_2 \le \Gamma_j .
\]
\[
    \tilde{\bm v}_{i|j}
    = \bm v_i - \tilde{\bm v}_j
    = (\bm v_i - \bar{\bm v}_j) - \bm Q_j \xi_j
    = \bar{\bm v}_{i|j} - \bm Q_j \xi_j .
\]

\medskip
\textbf{Projection onto the extreme ray $\bm r_1$.}
Assume the closest point on the VO boundary lies on the ray $\bm r_1$ with 
$\|\bm r_1\|_2 = 1$:
\[
    \operatorname{Proj}_{\mathrm{span}\{\bm r_1\}}(\tilde{\bm v}_{i|j})
    = (\tilde{\bm v}_{i|j}^\top \bm r_1)\,\bm r_1 .
\]

\medskip
\textbf{ORCA displacement definition.}
\[
\begin{aligned}
    \bm u_{i|j}(\xi_j)
    &= \operatorname{Proj}_{\mathrm{span}\{\bm r_1\}}(\tilde{\bm v}_{i|j})
       \;-\; \tilde{\bm v}_{i|j} \\[1mm]
    &= \bigl(\bar{\bm v}_{i|j}^\top \bm r_1\bigr)\bm r_1 
       \;-\; \bar{\bm v}_{i|j}
       \;+\; (I - \bm r_1 \bm r_1^\top)\bm Q_j \xi_j \\[1mm]
    &= (\bm r_1 \bm r_1^\top - I)(\bm v_i - \bar{\bm v}_j)
       \;+\; (I - \bm r_1 \bm r_1^\top)\bm Q_j \xi_j .
\end{aligned}
\]

\medskip
\textbf{Affine form:}
\[
    \bm u_{i|j}(\xi_j)
    =
    \underbrace{(\bm r_1 \bm r_1^\top - I)(\bm v_i - \bar{\bm v}_j)}_{\displaystyle \bar{\bm u}_j}
    \;+\;
    \underbrace{(I - \bm r_1 \bm r_1^\top)\bm Q_j}_{\displaystyle \bm U_j}\,\xi_j .
\]

\normalsize
\end{frame}
% ------------------------------------------------------

% ------------------------------------------------------
\begin{frame}{Ben-Tal--Nemirovski Ellipsoidal Form}
\footnotesize
the robust ORCA constraint for agent $i$ w.r.t.\ neighbor $j$ is \[ \bigl( \bm v - (\bm v_i + \tfrac12\, \bm u_{i|j}(\xi_j)) \bigr)^\top \bm n_{i|j} \;\ge\; 0, \qquad \forall\, \|\xi_j\|_2 \le \Gamma_j . \]
Define the augmented decision vector
\[
    \bm x := 
    \begin{bmatrix}
        \bm v \\ 1
    \end{bmatrix}.
\]

The uncertain coefficient vector is
\[
    \tilde{\bm a}_j(\xi_j)
    =
    \begin{bmatrix}
        -\,\bm n_{i|j} \\[1mm]
        \bm n_{i|j}^\top \bm v_i
        + \tfrac12\, \bm n_{i|j}^\top \bar{\bm u}_j
    \end{bmatrix}
    +
    \begin{bmatrix}
        \bm 0 \\[1mm]
        -\tfrac12\, \bm U_j^\top \bm n_{i|j}
    \end{bmatrix}
    \xi_j .
\]

Thus the robust constraint becomes
\[
    \tilde{\bm a}_j(\xi_j)^\top \bm x \;\le\; 0,
    \qquad
    \forall\|\xi_j\|_2 \le \Gamma_j.
\]

This matches the standard Ben-Tal--Nemirovski ellipsoidal form:
\[
    \tilde{\bm a}_j(\xi_j)
    = \bar{\bm a}_j + \bm B_j \xi_j,
    \qquad
    \|\xi_j\|_2 \le \Gamma_j.
\]

\end{frame}
% ------------------------------------------------------

% ------------------------------------------------------
\begin{frame}{the Other Case}

\small
When the closest point on the VO boundary does \textbf{not} lie on either extreme
rays, i.e., on the circular arc, the same
robust-optimization framework still applies.

\medskip
\textbf{Key idea: geometric relaxation.}
\begin{itemize}
    \item Select a \textbf{supporting hyperplane} of the VO boundary together 
          with its associated half-space that excludes the VO.
    \item Instead of projecting onto $\mathrm{span}\{\bm r_1\}$ (extreme-ray case),
          project the noisy relative velocity onto this supporting hyperplane.
    \item This again yields a \textbf{Ben-Tal--Nemirovski ellipsoidal form} and
          the same type of deterministic robust constraint.
\end{itemize}

\medskip
\textbf{How to find the supporting hyperplane.}
\begin{itemize}
    \item Use the \textbf{nominal relative velocity} $\bar{\bm v}_{i|j}$ to identify the closest point 
          on the VO boundary.
    \item The tangent at that boundary point defines the supporting hyperplane
          and thus the ORCA half-space.
\end{itemize}


\end{frame}
% ------------------------------------------------------


% ------------------------------------------------------
\begin{frame}{Effect of Position Uncertainty on the VO}
\small

\textbf{Position uncertainty model.}
\[
    \tilde{\bm p}_j = \bar{\bm p}_j + E_j \,\zeta_j,
    \qquad \|\zeta_j\|_2 \le \Gamma_j .
\]

\medskip
\textbf{Impact on the VO geometry.}
\begin{itemize}
    \item Each noise realization $\zeta_j$ perturbs the relative position
          and thus produces a \textbf{different VO}.
    \item The uncertain VO is the \textbf{union} of these perturbed cones.
    \item Because position noise only shifts or slightly rotates the VO apex
          and boundary rays, the union remains a \textbf{truncated cone}.
    \item Therefore, given the noise between the observation position and velocity are independent, the robust ORCA still works!
\end{itemize}

\end{frame}
% ------------------------------------------------------



% ------------------------------------------------------
\begin{frame}{Position Uncertainty and Complexity Preservation}
\small

\begin{itemize}
    \item Each neighbor $j$ contributes \textbf{one linear half-space},
          regardless of which VO face is active or how uncertainty enters.
    \item The robust counterpart remains the \textbf{same convex QP} as in
          deterministic ORCA.
    \item Per-agent computation stays \textbf{$O(|\mathcal{N}_i|)$},
          enabling real-time performance.
\end{itemize}


\end{frame}
% ------------------------------------------------------

\section{Experiment Results}
\begin{frame}{More samples can lead to worse performance for (PE)}
    \begin{columns}[c]
        \begin{column}{0.5\textwidth}
            \centering
            \animategraphics[loop,autoplay,poster=first,width=0.95\textwidth]{5}{../results/experiment/8_2_PE_10_animation_frame}{0}{65}\\[2mm]
            \small 10 samples
        \end{column}
        \begin{column}{0.5\textwidth}
            \centering
            \animategraphics[loop,autoplay,poster=first,width=0.95\textwidth]{5}{../results/experiment/8_2_PE_50_animation_frame}{0}{69}\\[2mm]
            \small 50 samples
        \end{column}
    \end{columns}

\end{frame}

\begin{frame}{More samples can lead to worse performance for (SAA)}
    \begin{columns}[c]
        \begin{column}{0.5\textwidth}
            \centering
            \animategraphics[loop,autoplay,poster=first,width=0.95\textwidth]{5}{../results/experiment/4_1_SAA_10_animation_frame}{0}{16}\\[2mm]
            \small 10 samples
        \end{column}
        \begin{column}{0.5\textwidth}
            \centering
            \animategraphics[loop,autoplay,poster=first,width=0.95\textwidth]{5}{../results/experiment/4_1_SAA_50_animation_frame}{0}{16}\\[2mm]
            \small 50 samples
        \end{column}
    \end{columns}
\end{frame}

\begin{frame}{Larger budget leads to more conservative behaviour (RO)}
    \begin{columns}[c]
        \begin{column}{0.5\textwidth}
            \centering
            \animategraphics[loop,autoplay,poster=first,width=0.95\textwidth]{5}{../results/experiment/4_1_RO_0.2_animation_frame}{0}{13}\\[2mm]
            \small $\Gamma = 0.2$
        \end{column}
        \begin{column}{0.5\textwidth}
            \centering
            \animategraphics[loop,autoplay,poster=first,width=0.95\textwidth]{5}{../results/experiment/4_1_RO_0.4_animation_frame}{0}{13}\\[2mm]
            \small $\Gamma = 0.4$
        \end{column}
    \end{columns}
\end{frame}

\begin{frame}{Larger budget leads to more conservative behaviour (RO)}
    \begin{columns}[c]
        \begin{column}{0.5\textwidth}
            \centering
            \animategraphics[loop,autoplay,poster=first,width=0.95\textwidth]{5}{../results/experiment/8_2_RO_0.2_animation_frame}{0}{46}\\[2mm]
            \small $\Gamma = 0.2$
        \end{column}
        \begin{column}{0.5\textwidth}
            \centering
            \animategraphics[loop,autoplay,poster=first,width=0.95\textwidth]{5}{../results/experiment/8_2_RO_0.4_animation_frame}{0}{68}\\[2mm]
            \small $\Gamma = 0.4$
        \end{column}
    \end{columns}
\end{frame}

\begin{frame}{Results Summary}
    \centering
    \begin{table}[]
    \tiny
    \caption{Steps needed to achieve the targets, number of collisions and number of relaxations for different methods on various instances. The best results for each instance and measure are highlighted in bold.}
\begin{tabular}{llrrrrrrr}
\toprule
\multirow{2}{*}{Instance} & \multirow{2}{*}{Measures} & \multicolumn{7}{c}{Methods} \\
\cmidrule{3-9}
  &   & PE\_1  & PE\_10 & PE\_50 & SAA\_10    & SAA\_50    & RO\_0.2 & RO\_0.4    \\
\midrule \midrule \addlinespace
\multirow{3}{*}{4\_1} & Steps & \textbf{13} & 14 & \textbf{13} & 17 & 17 & 14  & 14 \\
  & Collisions    & \textbf{0} & 3  & \textbf{0} & 3  & 3  & \textbf{0}  & \textbf{0} \\
  & Relaxations   & \textbf{1} & 4  & \textbf{1} & 15 & 16 & 4   & 3  \\
  \midrule
\multirow{3}{*}{4\_2} & Steps & \textbf{42} & 43 & \textbf{42} & \textbf{42} & 44 & \textbf{42}  & \textbf{42} \\
  & Collisions    & 6  & 14 & 10 & 4  & 2  & \textbf{0}  & \textbf{0} \\
  & Relaxations   & \textbf{0} & \textbf{0} & \textbf{0} & \textbf{0} & 1  & \textbf{0}  & \textbf{0} \\
  \midrule
\multirow{3}{*}{8\_1} & Steps & \textbf{45} & 50 & \textbf{45} & 49 & 48 & 7475    & 50 \\
  & Collisions    & 21 & 14 & 17 & 5  & 17 & \textbf{0}  & 2  \\
  & Relaxations   & 32 & 26 & 33 & 29 & 50 & \textbf{10} & 19 \\
  \midrule
\multirow{3}{*}{8\_2} & Steps & $\geq$10000  & 66 & 70 & $\geq$10000  & 54 & \textbf{47}  & 69 \\
  & Collisions    & 2  & 2  & 2  & \textbf{0} & \textbf{0} & \textbf{0}  & \textbf{0} \\
  & Relaxations   & 4  & 5  & \textbf{3} & 8  & 14 & 12  & 12 \\
  \midrule
\multirow{3}{*}{15\_1}    & Steps & \textbf{50} & 51 & 55 & 52 & 52 & 56  & $\geq$10000  \\
  & Collisions    & 44 & 30 & 53 & 53 & 34 & 24  & \textbf{0} \\
  & Relaxations   & 78 & 56 & 75 & 114    & 108    & 68  & \textbf{0}     \\
  \bottomrule
\end{tabular}
\end{table}
\end{frame}
    
\begin{frame}{Conclusion}
    \begin{itemize}
        \item We proposed a robust ORCA framework that accounts for both position and velocity uncertainty.
        \item The robust ORCA constraints can be expressed in closed-form deterministic equivalents, preserving the computational efficiency of ORCA.
        \item Experimental results demonstrate that the robust ORCA effectively reduces collisions under uncertainty while maintaining reasonable path efficiency.
    \end{itemize}
\end{frame}

% ------------------------------------------------------
\frame{\centerline{\Large Thank you!}}
% ------------------------------------------------------

\end{document}